\documentclass[dvipdfmx,twocolumn,10pt]{jsarticle}
\usepackage[top=10truemm,bottom=10truemm,left=10truemm,right=10truemm]{geometry}
\usepackage[dvipdfmx]{graphicx}
\usepackage{url}

\renewcommand{\figurename}{Fig.}
\newcommand{\Figref}[1]{Fig~\ref{#1}}

% あがき
\renewcommand{\baselinestretch}{0.75}
\setlength\abovecaptionskip{1truemm}
\setlength\belowcaptionskip{1truemm}
\makeatletter
%\def\section{\@startsection {section}{1}{\z@}{-2.5ex plus -1ex minus -.2ex}{2.5 ex plus .2ex}{\Large\bf}}
%\def\subsection{\@startsection {subsection}{1}{\z@}{0.5ex}{0.5 ex}{\large\bf}}
%\def\subsection{\@startsection {subsection}{1}{\z@}{-1.0ex plus -1ex minus -.1ex}{3.0 ex plus .2ex}{\Large\bf}}
%\def\subsubsection{\@startsection {subsubsection}{1}{\z@}{-2.5ex plus -1ex minus -.2ex}{.3 ex plus .2ex}{\large \bf $\spadesuit$ }}
\makeatother
%\makeatletter
%  \renewcomand{\section}{
%    \@startsection{section}{1}{\z@}
%    {0.4\Cvs}{0.1\Cvs}
%    {\nomalfont\large\headfont\raggedright}}
%\makeatother

\begin{document}
\pagestyle{empty}
\title{\Large 雷雲を想定した強制により生じる\\
巨大惑星表層流の数値計算}
%\title{\Large 
%雷雲を想定した強制により生じる巨大惑星表層流の数値計算
%}

\author{\large Ryoma Suzuki (Planetary and Space Group)}
%\author{\large 鈴木 綾馬 (惑星宇宙グループ)}
\date{}
\maketitle
\vspace{-0.2zh}
\section{Introduction}
%\section{はじめに}
\vspace{-0.5zh}
巨大惑星の東西風分布と極渦のレジームは特徴がある.
%系外惑星には多様な質量, 惑星半径, 公転半径のものが存在する(http://exoplanets.org/). 
%その中には地球程度の質量をもつ惑星も多く存在している. 
%それらの中には岩石を主成分とする地球型惑星も存在すると考えられる. 
%地球的な生命を想定すると, 惑星表層に水が存在する場合惑星の生命存在可能性が期待される. 
%惑星表層に水が存在する条件を考察することは重要であるので, 生命存在可能性の検討を念頭に置いた気候推定が行われている(Noda et al, 2017 など). 

%系外惑星の生命存在可能性を念頭に置いた気候推定の 1 つである Abe et al. (2011, 以下 AASZ2011) は, 
%陸惑星の気候に関する大気大循環モデル (GCM) 実験を行った.
%陸惑星とは, 地球に比べて表層に存在する水が極端に少ない惑星である.
%AASZ2011 によれば, 陸惑星は惑星表層を水に覆われた惑星(水惑星)よりも大きな太陽放射吸収量で惑星表層に液体の水を保持できる. 
%そして, 惑星に液体の水が存在できる太陽放射吸収量の閾値を超えると, 惑星表層の水が全て蒸発する完全蒸発状態が発生する. 

%本研究では, 陸惑星における完全蒸発状態の発生に関する考察を行った. 
%当初, 陸惑星において完全蒸発状態が発生する条件の自転軸傾斜$\cdot$自転角速度依存性を調査することを目指していた.
%しかし, AASZ2011 と同様の設定を用いた再現実験を実施したところ, 彼等が示した太陽放射吸収量の臨界値を越えても完全蒸発状態が発生しない可能性があることがわかった. 
%そのため, 完全蒸発状態が発生するのか再検討することを目的とした. 

\vspace{-0.2zh}
\section{Model and experimental settings}
%\section{モデル, 実験設定}
\vspace{-0.5zh}
GCM used in this study is DCPAM5(\url{http://www.gfd-dennou.org/library/dcpam/}). 
The basic equations are primitive equations. 
Radiation scheme for Earth's atmosphere is used. 
The model includes processes of vertical diffusion, cloud convection and large scale condensation. 
%本研究で用いたモデルは惑星大気大循環モデル DCPAM5(\url{http://www.gfd-dennou.org/library/dcpam/})である.
%基礎方程式はプリミティブ方程式である.
%放射過程では地球を想定した放射スキームを用いる. 
%その他に鉛直乱流過程, 積雲対流過程, 大規模凝結過程を考慮している. 

The horizontal resolution is T21, the number of vertical level is 26. 
The values of solar constant S are given as S=1365, 1900, 2400, 3600 $\rm{W/m^{2}}$. 
Both of obliquity and eccentricity are 0. 
The values of the Earth are given as planetary radius, planetary rotation rate, gravity and so on. 
For the surface condition, two kinds of conditions are used:
one is an aqua planet condition that all of the surface is covered with a swamp ocean, 
the other is a land planet condition that a bucket model is applied to all of the surface. 
Two initial states are used:
a statistical equilibrium state of an aqua planet experiment with S=1365 $\rm{W/m^{2}}$, 
and a runaway greenhouse non-equilibrium state obtained with S=2000 $\rm{W/m^{2}}$ in an aqua planet experiment. 
%水平解像度は T21, 鉛直層数は 26 とした.
%実験では太陽定数 S として S=1365, 1900, 2400, 3600 ${\rm W/m^{2}}$ を与えた.
%惑星の自転軸傾斜角, 及び離心率はともに 0 とし, 惑星半径, 自転角速度, 重力などは地球設定と同じ値を用いた. 
%惑星表面の扱いに関しては水が常に供給される水惑星を想定した swamp ocean 設定と土壌に無限の水が溜まる陸惑星を想定した bucket model 設定の 2 つを用いた. 
%初期状態として水惑星に S=1365 ${\rm W/m^{2}}$ を与えて 15 年積分した統計的平衡状態(湿潤平衡解), 水惑星に S=2000 $\rm{W/m^{2}}$ を与えて平均鉛直積算大気水蒸気量が 400 $\rm{kg/m^{2}}$ となった非統計的平衡状態(暴走解)の 2 つの状態を用いた. 

\vspace{-0.2zh}
\section{Results}
%\section{結果}
\vspace{-0.5zh}
In all of experiments, statistical equilibrium states are obtained because an equatorial region is dry and an equatorial radiation is not limited. 
Result of the land planet experiment with S=2400 $\rm{W/m^{2}}$ shows that almost all of water exist in soil and the complete evaporation does not appear (\Figref{timeseries}). 
Net insolation at 2000 days (figure not shown) reaches 450 $\rm{W/m^{2}}$ which exceeds the threshold value obtained by AASZ2011 for the appearance of the complete evaporation. 
In order to confirm the dependence of a initial state, experiment with the non-equilibrium state as a initial state is performed. 
In this experiment, the complete evaporation does not appear. 
In order to confirm whether the complete evaporation appears or not with increased a solar constant, the experiment with S=3600 $\rm{W/m^{2}}$ is performed. 
Also in this case, the complete evaporation does not appear. 
The value of net insolation exceeds the threshold value with 670 $\rm{W/m^{2}}$. 
%backet model 設定において, S=2400 ${\rm W/m^{2}}$, 初期状態として湿潤平衡解を与えた実験では, 惑星表層の水はほとんど土壌へ分配され, 完全蒸発状態は発生しなかった(\Figref{timeseries}).
%太陽放射吸収量(図は示さない)は 2000 日目の時点で 450 ${\rm W/m^{2}}$ となっており, AASZ2011 の完全蒸発状態が発生する閾値を超えていた. 
%また, 初期値依存性の確認のため実施した, 初期値として暴走解を与えた場合でも, 完全蒸発状態は発生しなかった. 
%S を増加させた場合完全蒸発状態が発生するのか確認するため実施した, S=3600 $\rm{W/m^{2}}$ の実験でも完全蒸発状態は発生しなかった.
%その場合, 太陽放射吸収量は閾値を大きく超えた約 670 $\rm{W/m^{2}}$ となっていた. 

In land planet experiments, the Hadley and Ferrel circulations appear, which are the circulation pattern similar to those of the Earth. 
As for the hydrologic cycle, precipitation occurs only in the region where evaporation occurs: 
the equatorial region and the polar region. 
Almost no horizontal advection of vapor occurred. 
Almost all surface water is localized in the polar region except for slight amount in the equatorial region. 
The reason why surface water retains in polar region is that temperature in the polar region is kept to be low values due to zero obliquity. 
%In the cases with increased S, 
%because of increasing temperature of the equator region planetary radiation was increased, then temperature of polar region kept low temperature and planet maintained statistical equilibrium state. %要編集
%Note that due to because obliquity was 0, tempreture of the polar region was not increased so there was the possibility maintained the statistical equilibrium state. %要編集
%大気循環としては, 地球的なハドレー循環とフェレル循環が見られる. 
%水循環としては, 降水は赤道と極域の水が蒸発した地点で起こり, 水蒸気の水平移流はほぼなく, 水は赤道に存在する微量以外極に局在化している. 
%これは極域の温度が低いためである. %要編集
%極域の温度が低い理由として, 自転軸傾斜角が 0 であることが挙げられる. %要編集
%また, 陸惑星では赤道域の温度が高くなり得るため, 赤道域の温度が高く極域の温度が低い状態で統計的平衡状態を維持できる. %要編集

\vspace{-0.2zh}
\section{Conclusion}
%\section{まとめ}
\vspace{-0.5zh}
The above results suggest that there is a possibility that water is retained in the surface with net insolation larger than the threshold value obtained previous study(AASZ2011). 
One of the reasons why the complete evaporation does not occur is that temperature in the polar region is kept to be low value. 
Further investigation is necessary for considering the appearance condition of the complete evaporation with experiment changing obliquity. 
%以上の結果は過去の研究で議論された太陽放射吸収量よりもかなり大きな値で水が保持される可能性を示唆するものである. 
%ただし, 本研究の実験で完全蒸発状態が発生しない原因の 1 つとしては極の温度が上がらず水が蒸発しないことが考えられる. 
%今後は自転軸傾斜角を変えた実験を行い完全蒸発状態が発生しない原因について更に考察する必要がある. 

\begin{figure}[h]
  \centering
  \includegraphics[width=0.45\textwidth,trim=220 100 220 330,clip]{./fig/land_S2400/LP_SoilMoist-VertIntQH2OVap_horimean_time0.0-56575.0.eps}
  \caption{
Horizontal mean water amount as a function of time for the land planet experiment with S=2400 $\rm{W/m^{2}}$. 
Abscissa is time [days]. 
Time at the leftside is 5475 day : initial day of the experiment. 
Solid line and broken line represent surface water in soil and vertically integrated water vapor, respectively. 
%水分量の全球平均値の時間変化. 
%横軸は時間[day] であり 5475 day (図の左端) が陸惑星実験を開始した時刻である. 
%土壌水分量が実線, 鉛直積分大気水蒸気量が破線である. 
}
  \label{timeseries}
\end{figure}

\end{document}

